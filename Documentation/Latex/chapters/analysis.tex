\section{Analysis Stage}
\subsection{Main Use Cases}
\begin{itemize}
	\item An \textit{unregistered user} can
	\begin{itemize}
		\item Sign up using a non-duplicate username and a custom password
	\end{itemize}
	\item A \textit{registered (not logged) user} can
	\begin{itemize}
		\item Login using his/her own credentials
	\end{itemize}
	
	\item A \textit{logged user} can
	\begin{itemize}
		\item Visualize his/her list of \textbf{contacts}
		\item Send a new \textbf{message} to the selected \textbf{contact}
		\item Send a new \textbf{message}  to a new \textbf{contact}
		\item If a \textbf{message}  is received, visualize it thanks to the real-time interface update
		\item Logout
\end{itemize}
	\item The \textit{system} should
	\begin{itemize}
		\item Correctly forward each \textbf{message} to the correct receiver
		\item Store \textbf{messages} whose destination is an offline user: those messages will be forwarded when the receiver is online again
		\item Store all the \textbf{message} histories and send them to the specific \textbf{user} each time he/she logs in
	\end{itemize}
\end{itemize}

\subsection{Size and Scope of the Application}
As cited in the Introduction (chapter 1), the application has been designed for working within limited entities/environments, for example among close friends attending the same University. 
This is mainly because the selected approaches and technologies (for more details see next chapters) are not very scalable and they are suitable for a limited number of nodes.
Anyway, the following properties are guaranteed:


\begin{itemize}
	\item No message can ever be lost, regardless the fact that the receiving user is online
	\item The application is totally OS-independent
	\item The GUI provides a user-friendly experience and makes application easy to use
	\item Within small clusters, the application ensures good performance
\end{itemize}

